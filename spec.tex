\documentclass{article}
\usepackage[utf8]{inputenc}
\usepackage[margin=0.7in]{geometry}

\title{Renderer volumetrického terénu}
\author{Eduard Hopfer}
\date{17.10.2022}

\begin{document}

\maketitle

\section*{Abstrakt}
Cieľom projektu bude renderovať volumetrický terén, teda nebude to iba výšková mapa, bude možné modelovat[!] rôzne previsy a jaskyne. 
Budem pracovať s hybridným modelom, s diskrétnou aj spojitou reprezentáciou. 
Základ bude diskrétny model z kvazi-voxelov, ktory potom technikami z [ref#1] vyhladim do spojitej polochy. 
Rendering bude real-time a na vysledny 3d model sa bude dat pozerat z 
roznych stran z pohladu lietajucej kamery.

\section*{Reprezentácia dát}
V diskretnom pripade bude terén reprezentovaný ako 2d grid materiálových stackov. Je to v podstate voxelový terén, kde bloky majú neuniformnú výšku. Táto reprezentácia je použitá v [ref#1], kde sú popísané nejaké jej výhody. Jednotlivé časti stackov budú mať priradený "materiál" ako kameň, hlina, voda, vzduch. Ten bude určovať, ako terén v každom bode vyzerá a pripadne fyzikálne správanie (vid Mozne rozšírenia).

V spojitom pripade bude teren reprezentovany ako implicitna plocha. 
Tuto plochu dostanem z diskretneho modelu techinkou popisanou v [ref#1].

\section*{Rendering}
Z diskretneho modelu viem priamo renderovat geometriu jednotlivych stackov. 
Na renderovanie spojitej implicitnej plochy pouzijem algoritmus marching cubes [ref#2]

\section*{Implementačné detaily}
Program by bol napísaný v jazyku Rust s pomocou OpenGL a GLFW [ref#3]. Na shadere použijem jazyk GLSL.
Takto stále budem mať low-level silu používania OpenGL ale dostanem typované API a výhody moderných programovacích jazykov.

\section*{Mozne rozšírenia}
Podla toho, ako budem stihat, alebo na doladenie obtiaznosti prace mi napadli taketo pripadne rozsirenia.

\begin{itemize}
    \item Proceduralne generovanie zobrazeneho terenu
    \item Export vygenerovaného modelu v nejakom rozumnom formáte, ktorý sa dá použiť aj mimo moju aplikáciu
    \begin{itemize}
        \item Import modelov
    \end{itemize}
    \item Editor diskretneho/implicitneho modelu
    \item Fyzikálne stabilizácia terénu popísaná v [ref#1]
\end{itemize}

\section*{Referencie}
\begin{enumerate}
    \item https://perso.liris.cnrs.fr/eric.galin/Articles/2009-arches.pdf
    \item https://en.wikipedia.org/wiki/Marching\_cubes
    \item https://docs.rs/luminance/latest/luminance/
\end{enumerate}

\end{document}
